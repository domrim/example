%% Beispiel-Präsentation
\documentclass{beamer} 

\title[Beamer-Vorlage]{Präsentation mit \LaTeX{} Beamer}
\subtitle{entsprechend den Gestaltungsrichtlinien vom 1. August 2020} 
\author[Pinkie Pie]{Pinkie Pie}

\date[8.\,12.\,2020]{8. Dezember 2020}


\begin{document}

\begin{frame}{Inhaltsverzeichnis}
\tableofcontents
\end{frame}

\section{Erster Abschnitt}

\subsection{Erster Unterabschnitt}
\begin{frame}{Blöcke}

\end{frame}
	  
\subsection{Zweiter Unterabschnitt}
\begin{frame}{Auflistungen}
	Text
	\begin{itemize}
		\item Auflistung\\ Umbruch
		\item Auflistung
		\begin{itemize}
			\item Auflistung
			\item Auflistung
		\end{itemize}
	\end{itemize}
\end{frame}

\section{Zweiter Abschnitt}

\begin{frame}
        Bei Frames ohne Titel wird die Kopfzeile nicht angezeigt, und  
    der freie Platz kann für Inhalte genutzt werden.
\end{frame}

\begin{frame}[plain]
    Bei Frames mit Option \texttt{[plain]} werden weder Kopf- noch Fußzeile angezeigt.
\end{frame}

\begin{frame}[t]{Beispielinhalt}
    Bei Frames mit Option \texttt{[t]} werden die Inhalte nicht vertikal zentriert, sondern an der Oberkante begonnen.
\end{frame}

\begin{frame}{Beispielinhalt: Literatur}
    Literaturzitat:
\end{frame}

\end{document}
